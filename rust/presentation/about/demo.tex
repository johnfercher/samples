\documentclass[aspectratio=169]{beamer}

\usetheme{metropolis}
\usepackage{appendixnumberbeamer}

\usepackage{booktabs}
\usepackage[scale=2]{ccicons}

\usepackage{pgfplots}

\usepackage{xspace}
\newcommand{\themename}{\textbf{\textsc{metropolis}}\xspace}

\usepackage[utf8]{inputenc}
\usepackage[T1]{fontenc}

\usepackage{listings}
\usepackage{xcolor, colortbl}
\usepackage{multicol}
\usepackage{chronology}

\usepackage[]{algorithm2e}

\definecolor{done}{rgb}{0.36, 0.72, 0.36}
\definecolor{doing}{rgb}{1.0, 0.8, 0.0}
\definecolor{scheduled}{rgb}{0.25, 0.54, 0.79}

\newcommand{\done}{\cellcolor{done}}
\newcommand{\scheduled}{\cellcolor{scheduled}}
\newcommand{\doing}{\cellcolor{doing}}

\bibliographystyle{unsrt}

\lstdefinestyle{sharpc}{language=[Sharp]C, frame=lr, rulecolor=\color{blue!80!black}}


\title{Eliminando Gargalos de Processamento \newline Utilizando Rust}
\date{}
\author{Johnathan Fercher}
%\institute{Braspag}
\titlegraphic{\hfill\includegraphics[height=2.0cm]{imgs/logo.png}}

\begin{document}
\maketitle

\begin{frame}{Sumário}
  \setbeamertemplate{section in toc}[sections numbered]
  \tableofcontents[hideallsubsections]
\end{frame}

\section{Introdução}
\begin{frame}{Introdução}
	\begin{multicols}{2}		
		\begin{center}
			\includegraphics[width=5cm]{imgs/rust3d.png}
		\end{center}
		\footnotesize
		\begin{itemize}
			\item Programming Language;
			\item Memory and thread-safe;
			\item Rust $\rightarrow$ LLVM $\rightarrow$ C $\rightarrow$ EXE;
			\item C-Bindings;
			\item Object-Oriented and Functional;
			\item Unit-tests and Package Manager;
			\item Interfaces and Generics;
			\item Without Garbage Collector;
		\end{itemize}	
	\end{multicols}
\end{frame}

\begin{frame}{Motivação}
	\begin{quote}
		\hspace{0.5cm}"O clock dos processadores dobra a cada 18 meses."
		
		\hspace{8.2cm}Lei de Moore, 1965.
	\end{quote}
\end{frame}

\begin{frame}{Motivação}
	\begin{center}
		\includegraphics[width=12.5cm]{imgs/cores-history.png}
	\end{center}
\end{frame}

\begin{frame}{Motivação}
	\begin{quote}
		"The way the processor industry is going, is to add more and more cores, but nobody knows how to program those things. I mean, two, yeah; four, not really; eight, forget it."
		
		\hspace{8.2cm}Steve Jobs, Apple.
	\end{quote}
\end{frame}

\begin{frame}{Motivação}
	Programação Paralela
	\begin{center}
		\includegraphics[width=11.5cm]{imgs/meme.png}
	\end{center}
\end{frame}

\begin{frame}{Motivação}
Problemas:
\begin{itemize}
	\item \textcolor{red}{Data races};
	\item Deadlock;
	\item \textcolor{red}{Use After Free};
	\item \textcolor{red}{Double Free};
\end{itemize}
\end{frame}

\begin{frame}{Motivação}
	\begin{center}
		\includegraphics[width=13.5cm]{imgs/bug.png}
	\end{center}
\end{frame}

\begin{frame}{História}
	\begin{chronology}[5]{2002}{2015}{80ex}[\textwidth]
		\event[2002]{2003}{C \& LLVM}
		\event{2006}{Graydon Hoare}
		\event{2009}{Mozilla}
		\event{2010}{Rust-c in Rust}
		\event{2011}{Rust-c using LLVM}
		\event{2012}{Alpha}
		\event{2013}{Cargo}
		\event{2015}{Stable}
	\end{chronology}
\end{frame}

\begin{frame}{Desenvolvimento}
	\begin{itemize}
		\item Licença MIT no Github;
		\item Duas versões: Stable e Nightly;
		\item Atualizações a cada 6 semanas;
		\item Processo de RFC;
		\item Quando uma RFC é aprovada ela é adicionada na versão Nightly;
		\item Após algum tempo em Nightly, ela pode ser adicionada na versão Stable, deixada de lado ou alterada;	
	\end{itemize}
\end{frame}

\section{Quem usa? E para que?}

\begin{frame}{Quem usa? E para que?}
	\begin{center}
		\includegraphics[width=4.5cm]{imgs/dropbox.png}	
		
		"Optimizing cloud file-storage."
	\end{center}
\end{frame}

\begin{frame}{Quem usa? E para que?}
	\begin{center}
		\includegraphics[width=4.5cm]{imgs/canonical.jpeg}	
		
		"Everything from server monitoring to middleware!"
	\end{center}
\end{frame}

\begin{frame}{Quem usa? E para que?}
	\begin{center}
		\includegraphics[width=4.5cm]{imgs/smartthings.png}	
		
		"Developing memory-safe embedded applications on our SmartThings Hub and supporting services in the cloud."
	\end{center}
\end{frame}

\begin{frame}{Quem usa? E para que?}
	\begin{center}
		\includegraphics[width=4.5cm]{imgs/mozilla.png}	
		
		"Building the Servo browser engine, integrating into Firefox, other projects."
	\end{center}
\end{frame}

\begin{frame}{Quem usa? E para que?}
	\begin{center}
		\includegraphics[width=4.5cm]{imgs/coursera.png}	
		
		"Programming Assignments in secured Docker containers."
	\end{center}
\end{frame}

\begin{frame}{Quem usa? E para que?}
	\begin{center}
		\includegraphics[width=2.2cm]{imgs/chef.png}	
		
		"Letting you develop, deploy and manage infrastructure, run-time environments and applications."
	\end{center}
\end{frame}

\begin{frame}{Quem usa? E para que?}
	\begin{center}
		\includegraphics[width=4.5cm]{imgs/atlassian.png}	
		
		"We use Rust in a service for analyzing petabytes of source code."
	\end{center}
\end{frame}

\begin{frame}{Quem usa? E para que?}
	\begin{center}
		\includegraphics[width=3.0cm]{imgs/npm.jpeg}	
		
		"Replacing C and rewriting performance-critical bottlenecks in the registry service architecture."
	\end{center}
\end{frame}

\begin{frame}{Quem usa? E para que?}
	\begin{center}
		\includegraphics[width=3.5cm]{imgs/habitat.png}	
		
		"Rust aid us to remove bottlenecks."
	\end{center}
\end{frame}

\begin{frame}{Quem usa? E para que?}
	\begin{itemize}
		\item No site oficial da linguagem há mais 123 empresas que deixaram claro que utilizam Rust em produção;
	\end{itemize}
\end{frame}

\section{Sugestões de regras do Velocity}

\begin{frame}{Provas de Conceito}
	\begin{center}
		\includegraphics[width=9.0cm]{imgs/battle.png}	
	\end{center}
\end{frame}


\begin{frame}{Velocity}
	Regras de repetição:
	\begin{itemize}
		\item 5 repetições de um \textbf{Cpf} em 10 minutos;
		\item 10 repetições de um \textbf{Cartão} em 2 dias;
		\item 2 repetições de um \textbf{Ip} em 5 minutos;
	\end{itemize}

	\begin{chronology}[5]{1}{10}{80ex}[\textwidth]
		\event{1}{\textbf{1 min:} 127.0.0.1}
		\event{4}{\textbf{4 min:} 127.0.0.1}
		\event{7}{\textbf{7 min:} 127.0.0.1}
		\event{8}{\textcolor{red}{\textbf{8 min:} 127.0.0.1}}
	\end{chronology}
\end{frame}

\begin{frame}{Algoritmo}
	\begin{algorithm}[H]
		var transactions = List<Transaction>()\;
		var quarantine = new DateTime()\;
		\vspace{0.2cm}
		\For{transaction \textbf{in} transactions}{
			\eIf{quarantine\_is\_active(quarantine, transaction)}{
				block(transaction)\;
				update(quarantine)\;
			}{
				\If{extrapolate\_rule(transaction)}{
					block(transaction)\;
					update(quarantine)\;
				}	
			}
		}
	\end{algorithm}
\end{frame}

\begin{frame}{Sobre os Benchmarks V1.0}	
	\textbf{Hardware: I7 8770 (6 núcleos + 6 threads), 8GB RAM DDR4, SSD;}
	
	\begin{itemize}
		\item Todas as versões representam o Velocity de forma exata;
		\item Não é utilizado agrupamento e nem programação funcional;
		\item C\# e Rust paralelisam o processamento;
		\item C\# e Rust realmente processaram 640 regras, o tempo de R é uma estimativa;
		\item R executou no banco SQL de monitoria;
	\end{itemize}
\end{frame}

\begin{frame}{Benchmark (v1.0)}
	\begin{figure}
		\begin{tikzpicture}
		\begin{axis}[
		mbarplot,
		ymin=0, ymax=10.0,
		%xlabel={Foo},
		ylabel={\scriptsize Tempo de Execução},
		width=0.9\textwidth,
		height=8cm,
		xticklabels={,,}
		]
		
		\addplot+[black] plot coordinates {(0, 10.0)};
		
		\legend{Rust (10m)}
		
		\end{axis}
		\end{tikzpicture}
	\end{figure}
\end{frame}

\begin{frame}{Benchmark (v1.0)}
	\begin{figure}
		\begin{tikzpicture}
		\begin{axis}[
		mbarplot,
		ymin=0, ymax=200.0,
		%xlabel={Foo},
		ylabel={\scriptsize Tempo de Execução},
		width=0.9\textwidth,
		height=8cm,
		xticklabels={,,}
		]
		
		%\addplot plot coordinates {(0, 30.0)};
		\addplot+[purple] plot coordinates {(0, 300.0)};
		\addplot+[black] plot coordinates {(0, 10.0)};
		
		\legend{C\# (300m), Rust (10m)}
		
		\end{axis}
		\end{tikzpicture}
	\end{figure}
\end{frame}

\begin{frame}{Benchmark (v1.0)}
	\begin{figure}
		\begin{tikzpicture}
		\begin{axis}[
		mbarplot,
		ymin=0, ymax=1280.0,
		%xlabel={Foo},
		ylabel={\scriptsize Tempo de Execução},
		width=0.9\textwidth,
		height=8cm,
		xticklabels={,,}
		]
		
		\addplot+[gray] plot coordinates {(0, 1280.0)};
		\addplot+[purple] plot coordinates {(0, 300.0)};
		\addplot+[black] plot coordinates {(0, 10.0)};
		
		\legend{R (1280m), C\# (300m), Rust (10m)}
		
		\end{axis}
		\end{tikzpicture}
	\end{figure}
\end{frame}

\begin{frame}{Sobre os Benchmarks V2.0}	
	\textbf{Hardware: I7 8770 (6 núcleos + 6 threads), 8GB RAM DDR4, SSD;}
	
	\begin{itemize}
		\item Todos os algoritmos utilizam agrupamento;
		\item Rust e Python utilizam programação funcional;
		\item Somente a versão em Rust simula exatamente o Velocity;
		\item Python utiliza a lib Numpy, que é feita em C, C++ e Fortran;
	\end{itemize}
\end{frame}

\begin{frame}{Benchmark (v2.0)}
\begin{figure}
	\begin{tikzpicture}
	\begin{axis}[
	mbarplot,
	ymin=0, ymax=44.0,
	%xlabel={Foo},
	ylabel={\scriptsize Tempo de Execução},
	width=0.9\textwidth,
	height=8cm,
	xticklabels={,,}
	]
	
	\addplot+[black] plot coordinates {(0, 44.0)};
	
	\legend{Rust (44s)}
	
	\end{axis}
	\end{tikzpicture}
\end{figure}
\end{frame}

\begin{frame}{Benchmark (v2.0)}
	\begin{figure}
		\begin{tikzpicture}
		\begin{axis}[
		mbarplot,
		ymin=0, ymax=56.0,
		%xlabel={Foo},
		ylabel={\scriptsize Tempo de Execução},
		width=0.9\textwidth,
		height=8cm,
		xticklabels={,,}
		]
		
		%\addplot plot coordinates {(0, 30.0)};
		\addplot+[orange] plot coordinates {(0, 56.0)};
		\addplot+[black] plot coordinates {(0, 44.0)};
		
		\legend{Python (56s), Rust (44s)}
		
		\end{axis}
		\end{tikzpicture}
	\end{figure}
\end{frame}

\begin{frame}{Benchmark (v2.0)}
	\begin{figure}
		\begin{tikzpicture}
		\begin{axis}[
		mbarplot,
		ymin=0, ymax=1800.0,
		%xlabel={Foo},
		ylabel={\scriptsize Tempo de Execução},
		width=0.9\textwidth,
		height=8cm,
		xticklabels={,,}
		]
		
		\addplot+[gray] plot coordinates {(0, 1800.0)};
		\addplot+[orange] plot coordinates {(0, 68.0)};
		\addplot+[black] plot coordinates {(0, 44.0)};
		
		\legend{SQL (1800s), Python (68s), Rust (44s)}
		
		\end{axis}
		\end{tikzpicture}
	\end{figure}
\end{frame}

\section{Mostre-me o código}

\section{Curiosidades}

\begin{frame}{Curiosidades}
	\begin{itemize}
		\item Em apenas 3 anos o gerenciador de pacotes de Rust ultrapassou os gerenciadores de R e Haskell;
		\item Em 2017 a Red Hat começou a dar suporte para Rust;
		\item Rust possui uma imagem oficial no Docker Hub;
		\item Em 2017 todas as IDEs da IntelliJ passaram a dar suporte para Rust;
		\item Amazon e Google estão contratando desenvolvedores Rust;
		\item Facebook e Github fazem parte dos Gold Sponsors da RustConf;
		\item Intel encoraja a utilização de Rust;
		\item Rust possui interoperabilidade com C, C++, Swift, Python e Node.JS;
	\end{itemize}
\end{frame}

\begin{frame}{Eventos}
	\begin{center}
		\includegraphics[width=4.0cm]{imgs/rustfest.png}
		\hspace{1cm}	
		\includegraphics[width=5.0cm]{imgs/rustconf.png}	
	\end{center}
\end{frame}

\begin{frame}{Sistema Operacional}
	\begin{center}
	\includegraphics[width=13.0cm]{imgs/redox.jpg}	
	\end{center}
\end{frame}

\section{Material de estudos}

\begin{frame}{Rust by Example}
	\includegraphics[width=15.0cm]{imgs/rust-by-example.png}	
\end{frame}

\begin{frame}{Rust - Concorrência e alta performance com segurança}
	\begin{center}
		\includegraphics[width=5.0cm]{imgs/casa-do-codigo-rust.jpg}	
	\end{center}
\end{frame}

\begin{frame}{Programming Rust}
	\begin{center}
		\includegraphics[width=12.0cm]{imgs/programming-rust.jpg}	
	\end{center}
	Livro mais popular do Safari Ebooks no mês de seu lançamento em Outubro de 2017
\end{frame}

\begin{frame}{Hands-On Functional Programming in Rust}
	\begin{center}
		\includegraphics[width=5.0cm]{imgs/hands-on-functional-programming-in-rust.jpg}	
	\end{center}
\end{frame}

\begin{frame}{Rust Fundamentals}
	\begin{center}
		\includegraphics[width=13.0cm]{imgs/pluralsight.png}	
	\end{center}
\end{frame}

\begin{frame}[standout]
	\begin{center}
		\includegraphics[width=2.5cm]{imgs/rustacean.png}
	\end{center}
  	Obrigado
  		
\end{frame}

\end{document}
